\documentclass{article}
\usepackage[a4paper, margin=0.5in]{geometry}

\title{3 Idiots Review}
\author{Prann Bansal 150510, Yash Srivastav 150839}

\begin{document}

\maketitle

3 Idiots is the perfect end to an exciting year for India. The three idiots, Rancchoddas Shyamaldas Chanchad (Aamir Khan), Raju Rastogi (Sharman Joshi) and Farhan Qureshi (R Madhavan), are perfect archetypes of the new age Indian who is essentially a non-conformist, questioning outmoded givens, choosing to live life on his own terms and chartering new roads that consciously skirt the rat race.\\
 The film begins with the entry of our threesome in the city's elite engineering college. It takes the first tryst with the mandatory ragging sessions which enunciate who the leader of the gang is going to be: new entrant Baba Rancchoddas, as his friends fondly call him. Rancho not only leads his friends through the maze of India's competitive, high-pressure, rote-heavy, illogical and almost cruel education system, he tutors them on several life mantras too. Like, running after excellence, not success; questioning not blindly accepting givens; inventing and experimenting in lieu of copying and cramming; and essentially following your heart's calling if you truly want to make a difference.\\
The high point of the film is the fact that director Rajkumar Hirani says so much, and more, without losing his sense of humour and the sheer lightness of being. The film is a laugh riot, despite being high on fundas. Certain sequences almost have you rolling in the aisle, like the ragging sequence, Omi's chamatkar/balatkar speech, the threesome's wedding crasher sequence, their mournful meal with Raju's mournful mum and Rancho's sundry demos to prove how Kareena has chosen the wrong guy for herself. Add to this, the strong emotional core of the film that makes gentle tugs, now and then, at your guts, and you have an almost perfect score. Hirani carries forward his simplistic `humanism alone works' philosophy of the Lage Raho Munnabhai series in 3 Idiots too, making it a warm and vivacious signature tune to 2009. The second half of the film does falter in parts, specially the child birth sequence, but it doesn't take long for the film to jump back on track.\\

For more info see \cite{indiatimes-article}
\bibliographystyle{unsrt}
\bibliography{3idiots}

\end{document}
